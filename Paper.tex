% Options for packages loaded elsewhere
\PassOptionsToPackage{unicode}{hyperref}
\PassOptionsToPackage{hyphens}{url}
\PassOptionsToPackage{dvipsnames,svgnames,x11names}{xcolor}
%
\documentclass[
  letterpaper,
  DIV=11,
  numbers=noendperiod]{scrartcl}

\usepackage{amsmath,amssymb}
\usepackage{iftex}
\ifPDFTeX
  \usepackage[T1]{fontenc}
  \usepackage[utf8]{inputenc}
  \usepackage{textcomp} % provide euro and other symbols
\else % if luatex or xetex
  \usepackage{unicode-math}
  \defaultfontfeatures{Scale=MatchLowercase}
  \defaultfontfeatures[\rmfamily]{Ligatures=TeX,Scale=1}
\fi
\usepackage{lmodern}
\ifPDFTeX\else  
    % xetex/luatex font selection
\fi
% Use upquote if available, for straight quotes in verbatim environments
\IfFileExists{upquote.sty}{\usepackage{upquote}}{}
\IfFileExists{microtype.sty}{% use microtype if available
  \usepackage[]{microtype}
  \UseMicrotypeSet[protrusion]{basicmath} % disable protrusion for tt fonts
}{}
\makeatletter
\@ifundefined{KOMAClassName}{% if non-KOMA class
  \IfFileExists{parskip.sty}{%
    \usepackage{parskip}
  }{% else
    \setlength{\parindent}{0pt}
    \setlength{\parskip}{6pt plus 2pt minus 1pt}}
}{% if KOMA class
  \KOMAoptions{parskip=half}}
\makeatother
\usepackage{xcolor}
\setlength{\emergencystretch}{3em} % prevent overfull lines
\setcounter{secnumdepth}{-\maxdimen} % remove section numbering
% Make \paragraph and \subparagraph free-standing
\ifx\paragraph\undefined\else
  \let\oldparagraph\paragraph
  \renewcommand{\paragraph}[1]{\oldparagraph{#1}\mbox{}}
\fi
\ifx\subparagraph\undefined\else
  \let\oldsubparagraph\subparagraph
  \renewcommand{\subparagraph}[1]{\oldsubparagraph{#1}\mbox{}}
\fi


\providecommand{\tightlist}{%
  \setlength{\itemsep}{0pt}\setlength{\parskip}{0pt}}\usepackage{longtable,booktabs,array}
\usepackage{calc} % for calculating minipage widths
% Correct order of tables after \paragraph or \subparagraph
\usepackage{etoolbox}
\makeatletter
\patchcmd\longtable{\par}{\if@noskipsec\mbox{}\fi\par}{}{}
\makeatother
% Allow footnotes in longtable head/foot
\IfFileExists{footnotehyper.sty}{\usepackage{footnotehyper}}{\usepackage{footnote}}
\makesavenoteenv{longtable}
\usepackage{graphicx}
\makeatletter
\def\maxwidth{\ifdim\Gin@nat@width>\linewidth\linewidth\else\Gin@nat@width\fi}
\def\maxheight{\ifdim\Gin@nat@height>\textheight\textheight\else\Gin@nat@height\fi}
\makeatother
% Scale images if necessary, so that they will not overflow the page
% margins by default, and it is still possible to overwrite the defaults
% using explicit options in \includegraphics[width, height, ...]{}
\setkeys{Gin}{width=\maxwidth,height=\maxheight,keepaspectratio}
% Set default figure placement to htbp
\makeatletter
\def\fps@figure{htbp}
\makeatother
% definitions for citeproc citations
\NewDocumentCommand\citeproctext{}{}
\NewDocumentCommand\citeproc{mm}{%
  \begingroup\def\citeproctext{#2}\cite{#1}\endgroup}
\makeatletter
 % allow citations to break across lines
 \let\@cite@ofmt\@firstofone
 % avoid brackets around text for \cite:
 \def\@biblabel#1{}
 \def\@cite#1#2{{#1\if@tempswa , #2\fi}}
\makeatother
\newlength{\cslhangindent}
\setlength{\cslhangindent}{1.5em}
\newlength{\csllabelwidth}
\setlength{\csllabelwidth}{3em}
\newenvironment{CSLReferences}[2] % #1 hanging-indent, #2 entry-spacing
 {\begin{list}{}{%
  \setlength{\itemindent}{0pt}
  \setlength{\leftmargin}{0pt}
  \setlength{\parsep}{0pt}
  % turn on hanging indent if param 1 is 1
  \ifodd #1
   \setlength{\leftmargin}{\cslhangindent}
   \setlength{\itemindent}{-1\cslhangindent}
  \fi
  % set entry spacing
  \setlength{\itemsep}{#2\baselineskip}}}
 {\end{list}}
\usepackage{calc}
\newcommand{\CSLBlock}[1]{\hfill\break\parbox[t]{\linewidth}{\strut\ignorespaces#1\strut}}
\newcommand{\CSLLeftMargin}[1]{\parbox[t]{\csllabelwidth}{\strut#1\strut}}
\newcommand{\CSLRightInline}[1]{\parbox[t]{\linewidth - \csllabelwidth}{\strut#1\strut}}
\newcommand{\CSLIndent}[1]{\hspace{\cslhangindent}#1}

\KOMAoption{captions}{tableheading}
\makeatletter
\@ifpackageloaded{caption}{}{\usepackage{caption}}
\AtBeginDocument{%
\ifdefined\contentsname
  \renewcommand*\contentsname{Table of contents}
\else
  \newcommand\contentsname{Table of contents}
\fi
\ifdefined\listfigurename
  \renewcommand*\listfigurename{List of Figures}
\else
  \newcommand\listfigurename{List of Figures}
\fi
\ifdefined\listtablename
  \renewcommand*\listtablename{List of Tables}
\else
  \newcommand\listtablename{List of Tables}
\fi
\ifdefined\figurename
  \renewcommand*\figurename{Figure}
\else
  \newcommand\figurename{Figure}
\fi
\ifdefined\tablename
  \renewcommand*\tablename{Table}
\else
  \newcommand\tablename{Table}
\fi
}
\@ifpackageloaded{float}{}{\usepackage{float}}
\floatstyle{ruled}
\@ifundefined{c@chapter}{\newfloat{codelisting}{h}{lop}}{\newfloat{codelisting}{h}{lop}[chapter]}
\floatname{codelisting}{Listing}
\newcommand*\listoflistings{\listof{codelisting}{List of Listings}}
\makeatother
\makeatletter
\makeatother
\makeatletter
\@ifpackageloaded{caption}{}{\usepackage{caption}}
\@ifpackageloaded{subcaption}{}{\usepackage{subcaption}}
\makeatother
\ifLuaTeX
  \usepackage{selnolig}  % disable illegal ligatures
\fi
\usepackage{bookmark}

\IfFileExists{xurl.sty}{\usepackage{xurl}}{} % add URL line breaks if available
\urlstyle{same} % disable monospaced font for URLs
\hypersetup{
  pdftitle={Data Error Impact Analysis},
  pdfauthor={Xiaoyu (Alice) Miao},
  colorlinks=true,
  linkcolor={blue},
  filecolor={Maroon},
  citecolor={Blue},
  urlcolor={Blue},
  pdfcreator={LaTeX via pandoc}}

\title{Data Error Impact Analysis}
\author{Xiaoyu (Alice) Miao}
\date{2024-04-03}

\begin{document}
\maketitle

To provide a comprehensive analysis of the described scenario, it's
crucial to delve into each step of the process, the errors introduced,
their impacts on the data, and how these can be addressed or mitigated
in future research endeavors. This deeper dive aims to illuminate the
complexities of data integrity and analysis within a research context.

\subsubsection{\texorpdfstring{\textbf{Theoretical
Framework}}{Theoretical Framework}}\label{theoretical-framework}

The scenario begins with an assumption: the true data generating process
is a Normal distribution with a mean of one and a standard deviation of
one. Theoretically, such a distribution should reflect a symmetric,
bell-shaped curve where approximately 68\% of values lie within one
standard deviation of the mean, 95\% within two, and 99.7\% within
three. This theoretical distribution underpins many statistical models
and assumptions, setting the stage for our expectations. Pipino, Lee,
and Wang (2002)

\subsubsection{Simulated Data Generation and
Errors}\label{simulated-data-generation-and-errors}

In the course of generating 1,000 observations from a Normal
distribution, the process encountered significant complications due to
both instrumental limitations and errors introduced during data
handling. The instrument used for data collection had a memory limit
that inadvertently led to the overwriting of the final 100 observations
with the first 100, thereby reducing the dataset's effective uniqueness
and independence. This limitation imposed a repetition bias, skewing the
analysis by disproportionately representing the characteristics of the
overwritten observations. Furthermore, the dataset's integrity was
further compromised by systematic errors introduced during the cleaning
process. A notable sign error, where negative values were mistakenly
converted to positive, distorted the data distribution, affecting both
the mean and variance of the dataset and leading to misleading
inferences about the distribution's true characteristics. Additionally,
a decimal place error significantly impacted values between 1 and 1.1,
inaccurately shifting their magnitude and thus disproportionately
influencing a specific subset of the data. This error not only altered
the distribution's overall shape but also its central tendency measures,
exacerbating the challenges in accurately interpreting the dataset's
statistical properties. Taylor (1997)

\subsubsection{\texorpdfstring{\textbf{Analysis and
Impact}}{Analysis and Impact}}\label{analysis-and-impact}

\paragraph{Initial Observations}\label{initial-observations}

The mean of the cleaned dataset was found to be significantly higher
than zero, an expected result given the nature of the data generation
process and the specified mean of one. However, the introduced errors
complicate this interpretation. The sign error artificially inflates the
dataset's mean by reducing the presence of negative values, which would
otherwise lower the mean. The decimal place error, though affecting a
smaller portion of the data, introduces a downward bias for those
values, potentially mitigating the inflation caused by the sign error to
some extent.

\paragraph{Statistical Testing}\label{statistical-testing}

The use of a one-sample t-test to evaluate the hypothesis that the mean
of the dataset is greater than 0 is problematic in this context. The
test assumes that the data are normally distributed and that
observations are independent and identically distributed---assumptions
violated by the introduced errors. The repetition of the first 100
observations due to the instrument's memory limit compromises the
assumption of independence, while the systematic errors affect the
distribution's normality. Greenland et al. (2016)

\subsubsection{\texorpdfstring{\textbf{Findings}}{Findings}}\label{findings}

The analysis revealed significant deviations from the expected behavior
of a normally distributed dataset with a mean of 1 and a standard
deviation of 1. The mean of the cleaned dataset was significantly
greater than 0, which, while supportive of the original hypothesis, is
misleading due to the introduced errors.

\subsubsection{Effects of the Issues}\label{effects-of-the-issues}

The introduction of errors through instrumental limitations and the
mishandling of data during the cleaning process profoundly compromised
the integrity and interpretiveness of the dataset. Specifically, the
overwriting of observations due to instrument memory constraints
resulted in a dataset that did not accurately represent the intended
sample of 1,000 unique observations, leading to a skewed analysis that
was not reflective of the true data generating process. Moreover, the
alteration of data values by the research assistant---converting
negative values to positive and incorrectly adjusting the decimal place
for a subset of the data---further exacerbated the issue. These actions
artificially inflated the mean and introduced a systematic bias,
undermining the validity of subsequent statistical tests and models. The
violation of key statistical assumptions, such as the independence and
identical distribution of data, alongside the introduction of biased
estimates for central tendency and variability, highlighted the
significant impact of these errors on the research findings, ultimately
misleading the conclusions drawn from the data.

\subsubsection{Next Steps}\label{next-steps}

To mitigate the risk of such issues in future research endeavors, a
multifaceted approach is necessary. Implementing routine data audits and
error-checking algorithms can serve as an initial line of defense,
flagging potential duplicates, outliers, and systematic biases for
further review. Employing robust statistical techniques that are less
sensitive to data anomalies, alongside establishing clear, documented
data cleaning protocols, can further safeguard the analysis against the
impacts of data integrity issues. Enhancing the training and awareness
of research personnel regarding common data errors and their
implications is crucial for preventing such issues from arising in the
first place. Additionally, incorporating data visualization techniques
as a regular component of the data cleaning and analysis process allows
for a more intuitive assessment of data quality, helping to identify
inconsistencies and anomalies that may not be evident through
statistical tests alone. Pre-analysis data validation steps, focusing on
the assessment of data quality and integrity, ensure that the dataset
accurately reflects the underlying phenomena of interest before any
major analyses are conducted. Together, these strategies form a
comprehensive approach to maintaining data integrity, ensuring that
research findings are both reliable and reflective of true
data-generating processes.

\subsubsection{\texorpdfstring{\textbf{Conclusion}}{Conclusion}}\label{conclusion}

The scenario highlights the multifaceted challenges of ensuring data
integrity and conducting accurate data analysis. The introduced errors
significantly impact the dataset's characteristics, leading to
potentially misleading conclusions. By implementing robust mitigation
strategies, researchers can enhance the reliability of their analyses.
These strategies, encompassing instrument checks, data auditing, error
correction protocols, statistical anomaly detection, transparency,
education, and sensitivity analyses, are crucial for safeguarding
against and addressing data integrity issues. Ultimately, such measures
ensure that research findings are reflective of the true
data-generating.

\Newpage

\#References

\phantomsection\label{refs}
\begin{CSLReferences}{1}{0}
\bibitem[\citeproctext]{ref-Greenland2016StatisticalMisinterpretations}
Greenland, Sander, Stephen Senn, Kenneth Rothman, John Carlin, Charles
Poole, Steven Goodman, and Douglas Altman. 2016. {``Statistical Tests, p
Values, Confidence Intervals, and Power: A Guide to
Misinterpretations.''} \emph{European Journal of Epidemiology} 31 (4):
337--50. \url{https://doi.org/10.1007/s10654-016-0149-3}.

\bibitem[\citeproctext]{ref-Pipino2002DataQuality}
Pipino, Leo L., Yang W. Lee, and Richard Y. Wang. 2002. {``Data Quality
Assessment.''} \emph{Communications of the ACM} 45 (4): 211--18.

\bibitem[\citeproctext]{ref-Taylor1997ErrorAnalysis}
Taylor, John R. 1997. \emph{An Introduction to Error Analysis: The Study
of Uncertainties in Physical Measurements}. University Science Books.

\end{CSLReferences}



\end{document}
